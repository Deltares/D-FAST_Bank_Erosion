\chapter{File formats} \label{Chp:FileFormats}

The software distinguishes 7 files:

\begin{itemize}
\item The \emph{analysis configuration file} defines the settings that are relevant for the execution of the software, it will point to other data files for bulk data.
\item The \emph{line geometry files} specify the coordinates of a line.
\item The \emph{chainage file} defines the river chainage along a line.
\item The \emph{parameter files} define optional spatial variations in the analysis input parameters.
\item The \emph{simulation result files} define the spatial variations in the velocities and water depths as needed by the algorithm.
\item The \emph{eroded volume file} reports on the eroded volumes along the analyzed river reach per chainage bin.
\item The \emph{dialog text file} defines all strings to be used in the interaction with the users (GUI, report, or error messages).
\end{itemize}

Each file type is addressed separately in the following subsections.

\section{analysis configuration file} \label{Sec:cfg}

The analysis configuration file of WAQBANK looked a lot like an ini-file, but deviated slightly due to different comment style and missing chapter blocks.
\dfastbe still supports the old files, but it's recommended to upgrade them to the newer file format described here which conforms to the usual ini-file format.
The user interface reads old (and obviously new) files, but always writes new files.

The file must contain a \keyw{[General]} block with a keyword \keyw{Version} to indicate the version number of the file.
The initial version number will be \keyw{1.0}.

Details on the other keywords supported are given below.
The order of the keywords in the file doesn't matter, but each block and each keyword should occur only once.

\begin{longtable}{l|l|p{8cm}}
%\begin{tabular}{l|l|p{8cm}}
Block & Keyword & Description \\ \hline
\keyw{General} & \keyw{Version} & Version number. Must be \keyw{1.0} \\
& \keyw{RiverKM} & Text file with river chainage (kilometres) and corresponding xy-coordinates \\
& \keyw{Boundaries} & River chainage of the region of interest specified as rkm-start:rkm-end, e.g. 81:100 (default: all) \\
& \keyw{BankDir} & Directory for storing bank lines (default: current directory) \\
& \keyw{BankFile} & Text file(s) in which xy-coordinates of bank lines are stored (default 'bankfile') \\
& \keyw{Plotting} & Flag indicating whether figures should be created (default: true) \\
& \keyw{SavePlots} & Flag indicating whether figures should be saved (default: true) \\
& \keyw{ClosePlots} & Flag indicating whether figures should be closed when the run ends (default: false) \\
& \keyw{FigureDir} & Directory for storing figures (default relative to work dir: figure) \\
& \keyw{FigureExt} & String indicating the file extension and type to be used for the figures saved (default: .png) \\

\keyw{Detect} & \keyw{SimFile} & Name of simulation output file to be used for determining representative bank line \\
& \keyw{WaterDepth} & Water depth used for defining bank line (default: 0.0) \\
& \keyw{NBank} & Number of bank lines, (default: 2) \\
& \keyw{Line<$i$>} & Textfile with xy-coordinates of search line \keyw{<$i$>} \\
& \keyw{DLines} & Distance from pre-defined lines used for determining bank lines (default: 50) \\

\keyw{Erosion} & \keyw{TErosion} & Simulation period \unitbrackets{years} \\
& \keyw{RiverAxis} & Textfile with xy-coordinates of river axis \\
& \keyw{Fairway} & Textfile with xy-coordinates of fairway axis \\
& \keyw{OutputInterval} & Bin size for which the eroded volume output is given (default: 1 km) \unitbrackets{km} \\
& \keyw{NLevel} & Number of discharge levels \\
& \keyw{RefLevel} & Reference level: discharge level with \keyw{SimFile<$i$>} that is equal to \keyw{SimFile} (only used when \keyw{Nlevel} > 1)  (default: 1) \\
& \keyw{SimFile<$i$>} & NetCDF map-file for computing bank erosion for discharge \keyw{<$i$>} (only used when \keyw{Nlevel} > 1) \\
& \keyw{PDischarge<$i$>} & Probability of discharge \keyw{<$i$>} (sum of probabilities should be 1) \\
& \keyw{OutputDir} & Directory for storing output files \\
& \keyw{BankNew} & Text file(s) in which new xy-coordinates of bank lines are stored (default 'banknew') \\
& \keyw{BankEq} & Text file(s) in which xy-coordinates of equilibrium bank lines are stored (default: 'bankeq') \\
& \keyw{EroVol} & Text file in which eroded volume per river-km is stored (default: 'erovol.evo') \\
& \keyw{ShipType} & Type of ship (per river-km) \\
& \keyw{Vship} & Relative velocity of the ships (per river-km) \unitbrackets{m/s} \\
& \keyw{Nship} & Number of ships per year (per river-km) \\
& \keyw{Nwave} & Number of waves per ship (default 5) \\
& \keyw{Draught} & Draught of the ships (per river-km) \unitbrackets{m} \\
& \keyw{Wave0} & Distance from fairway axis at which wave height is zero (default 200 m) \\
& \keyw{Wave1} & Distance from fairway axis at which reduction of wave height to zero starts (default Wave0-50 m) \\
& \keyw{Classes} & Use classes (true) or critical shear stress (false) in \keyw{BankType} (default: true) \\
& \keyw{BankType} & Bank strength definition (for each bank line per river-km) \\
& \keyw{ProtectLevel} & Text file(s) with level of bank protection for each bank line per river-km (default: -1000) \\
& \keyw{Slope} & Text file(s) with equilibrium slope for each bank line per river-km  (default: 20) \\
& \keyw{Reed} & Text file(s) with reed wave damping coefficient for each bank line per river-km  (default: 0) \\
& \keyw{VelFilter} & Filtering velocity along bank lines (default: true)
%\end{tabular}
\end{longtable}

\subsubsection*{Example}

\begin{Verbatim}
[General]
  Version        = 1.0
  RiverKM        = inputfiles\rivkm_20m.xyc
  Boundaries     = 68:230
  BankDir        = files\outputbanklines
  BankFile       = bankline

[Detect]
  SimFile        = inputfiles\SDS-krw3_00-q0075_map.nc
  WaterDepth     = 0.0
  NBank          = 2
  Line1          = inputfiles\oeverlijn_links_mod.xyc
  Line2          = inputfiles\oeverlijn_rechts_mod.xyc
  DLines         = [20,20]

[Erosion]
  TErosion       = 1
  RiverAxis      = inputfiles\maas_rivieras_mod.xyc
  Fairway        = inputfiles\maas_rivieras_mod.xyc
  NLevel         = 2
  RefLevel       = 1
  SimFile1       = inputfiles\SDS-krw3_00-q0075_map.nc
  PDischarge1    = 0.25
  SimFile2       = inputfiles\SDS-krw3_00-q1500_map.nc
  PDischarge2    = 0.75
  OutputDir      = files\outputbankerosion
  BankNew        = banknew
  BankEq         = bankeq
  EroVol         = erovol_standard.evo
  OutputInterval = 0.1
  ShipType       = 2
  Vship          = 5.0
  Nship          = inputfiles\nships_totaal
  Nwave          = 5
  Draught        = 1.2
  Wave0          = 150.0
  Wave1          = 110.0
  Classes        = false
  BankType       = inputfiles\bankstrength_tauc
  ProtectLevel   = inputfiles\stortsteen
\end{Verbatim}

\section{line geometry files}

This file specifies the coordinates of a single line.
It is used to specify

\begin{itemize}
\item RiverAxis
\item Fairway
\item Original or moved bank lines
\end{itemize}

The file format is equal to the file format used by WAQBANK.
It consists of two data columns: the first column specifies the x-coordinate and the second column the y-coordinate of each node of the line.

\subsubsection*{Example}

\begin{Verbatim}
1.1887425781300000e+005  4.1442128125000000e+005
1.1888840213863159e+005  4.1442400328947371e+005
1.1890254646426316e+005  4.1442672532894736e+005
1.1891669078989474e+005  4.1442944736842107e+005
1.1893083511552632e+005  4.1443216940789472e+005
1.1894497944115789e+005  4.1443489144736843e+005
1.1895912376678947e+005  4.1443761348684208e+005
1.1897326809242106e+005  4.1444033552631579e+005
1.1898741241805263e+005  4.1444305756578950e+005

...continued...
\end{Verbatim}

\section{chainage file}

This file defines the river chainage along a line.
The file format is equal to the file format used by WAQBANK.
It consists of three data columns: the first column specifies the chainage, the second and third columns specify the x- and y-coordinates of each node of the line.

\subsubsection*{Example}

\begin{Verbatim}
3     175908.078100      308044.062500
6     176913.890600      310727.781300
10     176886.578100     314661.750000
15     176927.328100     319589.687500
16     176357.000000     320335.375000 

...continued...
\end{Verbatim}

\section{parameter files}

These files may be used to define spatial variations in the input parameters needed by the analysis.
Many parameters may be varied along the analyzed river reach.

\begin{itemize}
\item wave0
\item wave1
\item vship
\item nship
\item nwave
\item draught
\item shiptype
\item slope
\item reed
\item banktype
\item protectlevel
\end{itemize}

The file format is independent of the parameter for which it's used.
The file format is equal to the file format used by WAQBANK.
It consists of two data columns: the first column specifies the chainage and the second column the value at that location.

\subsubsection*{Example}

\begin{Verbatim}
65.3    20912
67.6    18529
100.7   20375
146.8   24758
175.5   13911
201     15613
\end{Verbatim}

\section{simulation result files}

For the bank line detection and bank erosion analysis, the program  needs results from a hydrodynamic model.
The WAQBANK program was able to read data from the WAQUA SDS-output files or the Delft3D-FLOW trim-files.
\dfastbe now supports the results of \dflowfm in netCDF map-files following UGRID conventions.
The model needs the following data fields:

\begin{itemize}
\item x-coordinates of the mesh nodes
\item y-coordinates of the mesh nodes
\item face\_node\_connectivity of the mesh
\item bed levels zb at the mesh nodes
\item water levels zw at the mesh faces
\item water depths h at the mesh faces
\item velocity vector (ucx,ucy) at the mesh faces
\end{itemize}

The simulation result files may contain multiple time steps; the final time steps will be used for the analysis.


\section{eroded volume file}

This file reports on the eroded volumes per bank along the analyzed river reach per user defined chainage bin.
The file consists $1+N$ tab-separated data columns where $N$ is the number of bank lines processed: the first column specifies the chainage and the other columns report on the eroded bank volume per bank line accumulated be chainage bin.
The chainage coordinate provided is the upper limit of the chainage bin for which the volume is reported on that line.
The file format differs slightly from the file format used by WAQBANK since that file contained $N$ identical chainage columns followed by the $N$ eroded volume columns.

\subsubsection*{Example}

\begin{Verbatim}
68.00   2.21    0.00
68.10   6.44    0.00
68.20   7.81    0.00
68.30   43.63   161.39
68.40   14.24   0.00
68.50   8.88    0.00
68.60   0.00    0.00
68.70   0.00    0.00
68.80   2.39    0.00
68.90   0.00    0.00
69.00   0.88    0.00
69.10   7.40    69.27
69.20   5.64    65.47
69.30   11.98   55.78

...continued...
\end{Verbatim}

\section{dialog text file}

The dialog text file uses the block labels enclosed by square brackets of the common ini-file format, but the lines in between the blocks are treated verbatim and don't list keyword/value pairs.
Every print statement in the program is associated with a short descriptive identifier.
These identifiers show up in the dialog text file as the block labels.
The text that follows the block label will be used at that location in the program.
The order of the blocks in the file is not important.
Please note that every line is used as is, so don't add indentations or blank lines unless you want those to show up during the program execution.
Most blocks may contain any number of lines, but some blocks may only contain a single line.
Some data blocks may contain one or more Python-style placeholders used for inserting values.

\subsubsection*{Example}

The following excerpt of the default \keyw{messages.UK.cfg} dialog text file shows the string definition for 3 identifiers, namely \keyw{''} (the identifier for an empty line), \keyw{'header\_banklines'} and \keyw{'end\_banklines'}.
The header string contains two placeholders, namely \keyw{\{version\}} for the the version number and \keyw{\{location\}} for the url to the source code location.

\begin{Verbatim}
[]

[header_banklines]
=====================================================
Determine bank lines
=====================================================
version: {version}
source: {location}
-----------------------------------------------------
[end_banklines]

=====================================================
===        BankLines ended successfully!          ===
=====================================================
... continued ...
\end{Verbatim}