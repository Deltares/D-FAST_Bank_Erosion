\chapter{Introduction} \label{Chp:Introduction}

Since 2010 Deltares has been working on various bank erosion modules that can be used in combination with \dflow, WAQUA and \dflowfm.
The \dfastbe tool described in this document is the \dflowfm compatible successor of WAQBANK developed for WAQUA and \dflow.
The module computes local erosion sensitivity, visualizes the bank movement and gives an estimation of the amount of bank material that is released during a specified period (typically 1 year) and until an equilibrium is reached.
\dfastbe can easily be used as a post processing tool of \dflowfm simulations.
Some examples for which \dfastbe can compute bank erosion are:

\begin{itemize}
\item Bank adjustments such as removal of bank protection
\item Applying or removing fore shore protection
\item Changes in shipping type or intensity
\item Changes in currents (e.g. due to construction of side channels)
\end{itemize}

\textbf{Input}

The input of \dfastbe is:

\begin{enumerate}
\item \dflowfm results (netCDF map-files)
\item Initial bank lines or search lines for detecting bank lines
\item Local data: level of bank protection removal, subsoil characteristics, shipping information (quantity, location of fairway)
\end{enumerate}

\textbf{Output}

The output of \dfastbe is:

\begin{enumerate}
\item The new bank line position after specified period, and the equilibrium position.
\item Amount of sediment that is released from the banks over the specified period, and until the final equilibrium state is reached.
\end{enumerate}

This output is presented in graphs / figures and written to text files.

\section{\dfastbe versus WAQBANK}

\dfastbe will properly read the old analysis definition files of WAQBANK, but the new program is not able to read data directly from the WAQUA and \dflow output files.
These output files need to be converted to netCDF files that mimic the map-files written by \dflowfm; a small MATLAB based conversion 'sim2ugrid' is available for that purpose.
Since the old WAQBANK code was quite slow, intermediate files were stored in local folders for caching.
Since the algorithms of the new \dfastbe program are much faster, the caching step has been removed and \keyw{LocalDir} is ignored
