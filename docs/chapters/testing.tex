\chapter{Test Plan} \label{Chp:TestPlan}

The results of the software is verified by means of

\begin{itemize}
\item Unit testing at the level of functions, such as reading and writing of files, and basic testing of the algorithms.
Currently about half of the functions included in \keyw{io.py} are covered by unit tests; more to be added.
These tests are carried out by means of the \keyw{pytest} framework.
\item Regression tests have to configured to verify that the results of the batch mode remain unchanged under further code developments.
\end{itemize}

For the verification of \dfastbe three sets of input files for the Meuse River have been used:

\begin{enumerate}
\item One set of WAQUA result files converted using \keyw{sim2ugrid.m} to \dflowfm like netCDF files.
\item One set of \dflowfm result files of simulations using the same curvilinear mesh as was used in WAQUA.
\item One set of \dflowfm result files of simulations using a new unstructured mesh.
\end{enumerate}

Each configuration gives slightly different results; for further regression testing in particular the first and last one will be used.
For the automated testing, unit tests and regression tests based on known input/output combinations will be used.
These tests will be executed on the original Python code (i.e. in source code form) and to the degree possible on the compiled binaries as well.

\section{Acceptance testing}

In \autoref{Sec:FuncReq} the 10 functional requirements were listed.
They are repeated below and for every requirement it is indicated how it is tested.

\begin{enumerate}
\item The results of \dfastbe must match those of WAQBANK given the same input data.
This is tested by means of a number of comparison studies, and subsequently tested using regression tests.

\item Users must be able to run this program in batch mode from the command line.
This has been implemented as run modes \keyw{--mode banklines} and \keyw{--mode bankerosion}.
The proper functioning of these options is tested by means of regression testing.

\item Users must be able to run the analysis based on \dflowfm results.
This is tested by means of regression testing using result files of \dflowfm version X.

\item Users must be able to provide all data via an input file, similar to the ini-file like file of WAQBANK.
This testing is included in the regression testing of the batch mode, and may in the future be included in the gui testing.

\item The input files must be consistent with those of WAQBANK, or aligned with open standards or the \dflowfm modeling system.
The \dfastbe configuration file is identical to the WAQBANK definition file except for the fact that the new file contains three sections \keyw{[General]}, \keyw{[Detect]} and \keyw{[Erosion]} to give a bit more context for the purpose of each keyword.
\dfastbe accepts old WAQBANK input files, but writes the data in the new format.
The format of the other input files is not adjusted, but for line geometries \dfastbe also accepts shape files besides the original \file{.xyc} files.
There are unit and integration tests addressing the reading of the input files.

\item The output files must be consistent with those of WAQBANK, or aligned with open standards or the \dflowfm modeling system.
The ASCII files containing the bank erosion volumes are identical to those of WAQBANK.
The shifted bank lines are exported as Shape files instead of \file{.xyc} files to align with common GIS standards.
The figures are now saved as \file{.png} files instead of MATLAB specific \file{.fig} files.
There are unit and integration tests addressing the writing of the output files.

\item The should read relevant data directly from \dflowfm map-files similarly to WAQBANK reading data directly from SIMONA and Delft3D 4 result files.
All quantities previously read from the SIMONA SDS-files and Delft3D-FLOW trim-files is now read from the \dflowfm map.nc files.

\item A simple graphical user interface could support users in process of creating the input file.
The graphical user interface that you get by running \dfastbe in default mode or by explicitly specifying \keyw{--mode gui} has been tested manually as described in \autoref{Sec:GuiTests}.

\item It would be nice if the software would be more generally applicable than just the Dutch rivers.
The code does not include specific knowledge of the Dutch rivers except for the fact that some of the rules of thumb have been derived using Dutch river data.
The original WAQBANK code was already applied to foreign rivers such as the Danube.
No special testing carried out for this requirement.

\item It would be nice if the software would be able to run besides English also in Dutch.
All texts shown by \dfastbe are read from a language configuration file.
An English and a Dutch version of that configuration file are provided.
A most system tests are carried out using the default English configuration, but one test is carried out using the Dutch configuration.
\end{enumerate}

\section{System testing}

The whole system is tested via the command line (entry via \keyw{\_\_main\_\_}) and via Python calls to the \keyw{run} function in the \keyw{cmd} module.
These tests are repeated for the standalone compiled \dfastbe executable.
For the system testing a limited number of regression tests are carried out comparing the latest results against previously accepted results.

Since the testing of the graphical user interface is not included in the automated testing, a test protocol for manual tests has been defined.
These tests are described in the following section.

\subsection{Manual testing of the user interface} \label{Sec:GuiTests}

\subsubsection{Test 1: starting blank}
\subsubsection{Test 2: save default configuration file}
\subsubsection{Test 3: modify general settings}
\subsubsection{Test 4: modify detection settings}
\subsubsection{Test 5: modify erosion settings}
\subsubsection{Test 6: save modified configuration file}
\subsubsection{Test 7: load default configuration file}
\subsubsection{Test 8: load modified configuration file}
\subsubsection{Test 9: run detection analysis}
\subsubsection{Test 10: run erosion analysis}
\subsubsection{Test 11: view manual and about Windows}

\section{Integration testing}

The \keyw{batch} module builds on the functionality of the \keyw{io}, \keyw{support}, \keyw{kernel} and \keyw{plotting} modules to provide the bank line detection and erosion analysis functionality as main routines.
The main routines \keyw{banklines} and \keyw{bankerosion} are tested at this level via regression tests, the other routines in \keyw{batch} are tested as part of the unit testing.

\section{Unit testing}

Since the modules \keyw{io}, \keyw{support}, \keyw{kernel} and \keyw{plotting} only depend on third party modules, they are ideally suited for unit testing.
Most of the routines in \keyw{batch} can be addressed by means of unit testing as well.
An initial number of unit tests have been set up for \keyw{io} routines, more unit tests should be added.

All routines inside \keyw{gui} are interacting with the graphical user interface (either by extracting data from the state of the controls, or by setting the state of the controls).
These routines cannot be tested using the pytest framework, and are therefore tested by means of the manual system tests.

\section{Automated testing}

Automated TeamCity projects will be set up for testing the Python code, for building (and optionally signing of) binaries, and testing of the binaries.
In this way the formal release process can be easily aligned with the other products.
This is ongoing work; the test and build steps are currently run locally.

For all automated unit and regression tests the pytest framework is used.

\begin{Verbatim}
    > conda install pytest
    > pytest
\end{Verbatim}

%-------------------------------
\chapter{Test Report} \label{Chp:TestReport}

The test plan describes manual tests (the graphical user interface tests as part of the system testing) and automated testing (unit and regression testing).

\section{Manual tests}

This section summarizes the results of the manual testing of the graphical user interface.

\begin{tabular}{ll|l}
Test & Description & Success / Comments \\ \hline
1 & starting blank & OK \\
2 & save default configuration file & OK \\
3 & modify general settings & OK \\
4 & modify detection settings & OK \\
5 & modify erosion settings & OK \\
6 & save modified configuration file & OK \\
7 & load default configuration file & OK \\
8 & load modified configuration file & OK \\
9 & run detection analysis & OK \\
10 & run erosion analysis & OK \\
11 & view manual and about Windows & OK \\
\end{tabular}

\section{Automated tests}

Below a brief pytest report of the automated regression and unit testing is included.
Each period following a module name represents one successful test; failed test would be indicated by an F and a subsequent error report.

\begin{Verbatim}[fontsize=\tiny]
=============================================== test session starts ================================================
platform win32 -- Python 3.8.5, pytest-6.1.2, py-1.9.0, pluggy-0.13.1
rootdir: D:\checkouts\D-FAST\D-FAST_Bank_Erosion
collected 33 items

tests\test_io.py .................................                                                            [100%]

=============================================== 33 passed in 11.79s ================================================
\end{Verbatim}
