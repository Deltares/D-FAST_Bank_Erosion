\chapter{Testing} \label{Chp:Test}

\section{Test plan}

The testing is based on the pytest framework.

\begin{Verbatim}
    > conda install pytest
    > pytest
\end{Verbatim}

\subsection{Acceptance testing}

In \autoref{Sec:FuncReq} the 10 functional requirements were listed.
They are repeated below and for every requirement it is indicated how it is tested.

\begin{enumerate}
\item The results of \dfastbe must match those of WAQBANK given the same input data.
This is tested by means of a number of comparison studies, and subsequently tested using regression tests.

\item Users must be able to run this program in batch mode from the command line.
This has been implemented as run modes \keyw{--mode banklines} and \keyw{--mode bankerosion}.
The proper functioning of these options is tested by means of regression testing.

\item Users must be able to run the analysis based on \dflowfm results.
This is tested by means of regression testing using result files of \dflowfm version X.

\item Users must be able to provide all data via an input file, similar to the ini-file like file of WAQBANK.
This testing is included in the regression testing of the batch mode, and may in the future be included in the gui testing.

\item The input files must be consistent with those of WAQBANK, or aligned with open standards or the \dflowfm modeling system.
The \dfastbe configuration file is identical to the WAQBANK definition file except for the fact that the new file contains three sections \keyw{[General]}, \keyw{[Detect]} and \keyw{[Erosion]} to give a bit more context for the purpose of each keyword.
\dfastbe accepts old WAQBANK input files, but writes the data in the new format.
The format of the other input files is not adjusted, but for line geometries \dfastbe also accepts shape files besides the original \file{.xyc} files.
There are unit and integration tests addressing the reading of the input files.

\item The output files must be consistent with those of WAQBANK, or aligned with open standards or the \dflowfm modeling system.
The ascii files containing the bank erosion volumes are identical to those of WAQBANK.
The shifted bank lines are exported as Shape files instead of \file{.xyc} files to align with common GIS standards.
The figures are now saved as \file{.png} files instead of MATLAB specific \file{.fig} files.
There are unit and integration tests addressing the writing of the output files.

\item The should read relevant data directly from \dflowfm map-files similarly to WAQBANK reading data directly from SIMONA and Delft3D 4 result files.
All quantities previously read from the SIMONA SDS-files and Delft3D-FLOW trim-files is now read from the \dflowfm map.nc files.

\item A simple graphical user interface could support users in process of creating the input file.
The graphical user interface that you get by running \dfastbe in default mode or by explicitly specifying \keyw{--mode gui} has been tested manually as described in \autoref{Sec:GuiTests}.

\item It would be nice if the software would be more generally applicable than just the Dutch rivers.
The code does not include specific knowledge of the Dutch rivers except for the fact that some of the rules of thumb have been derived using Dutch river data.
The original WAQBANK code was already applied to foreign rivers such as the Donau.
No special testing carried out for this requirement.

\item It would be nice if the software would be able to run besides English also in Dutch.
All texts shown by \dfastbe are read from a language configuration file.
An English and a Dutch version of that configuration file are provided.
A most system tests are carried out using the default English configuration, but one test is carried out using the Dutch configuration.

\end{enumerate}

\subsubsection{Manual testing of the user interface} \label{Sec:GuiTests}

\subsubsection{Test 1}

\subsection{System testing}

\subsection{Integration testing}

\subsection{Unit testing}

\section{Test report}

Automated TeamCity projects will be set up for testing the Python code, for building (and optionally signing of) binaries, and testing of the binaries.
In this way the formal release process can be easily aligned with the other products.
This is ongoing work; the test and build steps are currently run locally

\begin{Verbatim}[fontsize=\tiny]
=============================================== test session starts ================================================
platform win32 -- Python 3.8.5, pytest-6.1.2, py-1.9.0, pluggy-0.13.1
rootdir: D:\checkouts\D-FAST\D-FAST_Bank_Erosion
collected 33 items

tests\test_io.py .................................                                                            [100%]

=============================================== 33 passed in 11.79s ================================================
\end{Verbatim}

The results of the software is verified by means of

\begin{itemize}
\item Unit testing at the level of functions, such as reading and writing of files, and basic testing of the algorithms.
All functions included in \keyw{io.py} and \keyw{kernel.py} are covered by unit tests.
These tests are carried out by means of the \keyw{pytest} framework.
\item Regression tests have been set up to verify that the results of the command line interactive mode (with redirected standard in input for files coming from WAQUA) and the batch mode (with configuration file input for files coming from either WAQUA or \dflowfm) remain unchanged under further code developments.
\end{itemize}

For the regression tests three sets of input files have been selected:

\begin{enumerate}
\item Convert one or two sets of legacy input files (SIMONA and/or Delft3D 4) to \dflowfm like netCDF files.
Running \dfastbe on those converted files should give results that are very similar to those obtained from a WAQBANK run on the original files.
\item Run \dflowfm simulations using the same curvilinear mesh as was used in WAQUA/Delft3D 4.
Running \dfastbe on the new files will give different results than those obtained from the WAQUA/Delft3D 4 results since a different hydrodynamic solver was used, but the differences are expected to be small.
They will be quantified and reported.
\item Run \dflowfm simulations using a new unstructured mesh.
Running \dfastbe on those new unstructured model results will give different results than those obtained using the curvilinear model, but the differences are expected to be small.
They will be quantified and reported.
\end{enumerate}

For the automated testing, unit tests and regression tests based on known input/output combinations will be used.
These tests will be executed on the original Python code and to the degree possible on the compiled binaries as well.
Details of the various tests implemented will be documented as the project progresses and full documentation will be included in the final project documentation.
