\chapter{Bank strength coefficients for different soil types} \label{bankcomp}

The strength coefficient of the bank material, $c_E$ , and the critical shear stress for erosion, $\tau_c$, for different soil types are given in \autoref{Tab5.1} based on \citep{VerheijMKSV95, Vergeer96, VerheijKNS98, Vroeg99}.

The following relation can be used

\begin{equation}
c_E = \frac{\beta}{\tau_c}
\end{equation}

where $\beta$ = $1.85 \cdot 10^{-4}$ \unitbrackets{kg / (m\textsuperscript{2} s\textsuperscript{3})}.

\begin{table}[H]
\center
\begin{tabular}{p{5cm}rr}
Bank type & $c_E$ \unitbrackets{m\textsuperscript{-1}s\textsuperscript{-1}} & $\tau_c$ \unitbrackets{Pa} \\ \hline
Protected bank & 0 & $\infty$ \\
Sturdy grass & 0,01 $10^{-4}$ & 185 \\
Mediocre grass & 0,02 $10^{-4}$ & 93 \\
Bad grass & 0,03 $10^{-4}$ & 62 \\
Very good clay (compact) & 0,5 $10^{-4}$ & 4 \\
Clay with 60\% sand (firm) & 0,6 $10^{-4}$ & 2,5 \\
Good clay with  little structure & 0,75 $10^{-4}$ & 2 \\
Strongly structure good clay (mediocre) & 1,5 $10^{-4}$ & 1,5 \\
Bad clay (weak) & 3,5 $10^{-4}$ & 0,65 \\
Sand with 17\% silt & 10 $10^{-4}$ & 0,20 \\
Sand with 10\% silt & 12,5 $10^{-4}$ & 0,15 \\
Sand with 0\% silt & 15 $10^{-4}$ & 0,10 \\ \hline
\end{tabular}
\caption{Corresponding values for $c_E$ and $\tau_c$ \citep{VerheijMKSV95}.}
\label{Tab5.1}
\end{table}
