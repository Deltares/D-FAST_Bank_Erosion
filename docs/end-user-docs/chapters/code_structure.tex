\chapter{Code structure}

\dfastbe code is subdivided into 7 files:

\begin{itemize}
	\item \file{\_\_init\_\_.py} module level file containing mainly the version number.
	\item \file{\_\_main\_\_.py} module level file containing argument parsing and call to \keyw{dfastbe.cmd.run()}.
	\item \file{base\_calculator.py} defining a parent \keyw{BaseCalculator} class for bank line detection and erosion steps.
	\item \file{cmd.py} containing the main run routine.
	\item \file{plotting.py} containing routines to visualize the results.
	\item \file{resources.py} is used to link to the \dfastbe logo.
	\item \file{utils.py} containing some utility functions.
\end{itemize}

and 4 sub-packages:

\begin{itemize}
	\item \file{bank\_lines} implements the workflow for the bank line detection analysis (triggered either by a \keyw{-{}-mode banklines} call or by clicking the \button{Detect} button in the graphical user interface).
	\item \file{bank\_erosion} implements the workflow for the bank line erosion analysis (triggered either by a \keyw{-{}-mode bankerosion} call or by clicking the \button{Compute} button in the graphical user interface).
	\item \file{gui} contains all functionality related to the graphical user interface.
	\item \file{io} contains the reading, parsing and writing of all configuration and result files.
\end{itemize}

The detailed description of the code per class and routine is beyond the scope of this document.
For more details the reader is referred to the documentation strings included in the source code.
