\chapter{File formats} \label{Chp:FileFormats}

The software distinguishes 9 files:

\begin{itemize}
\item The \emph{analysis configuration file} defines the settings that are relevant for the execution of the software, it will point to other data files for bulk data.
\item The \emph{line geometry files} specify the coordinates of a line.
\item The \emph{chainage file} defines the river chainage along a line.
\item The \emph{parameter files} define optional spatial variations in the analysis input parameters.
\item The \emph{simulation result files} define the spatial variations in the velocities and water depths as needed by the algorithm.
\item The \emph{eroded volume file} reports on the eroded volumes along the analyzed river reach per chainage bin.
\item The \emph{dialog text file} defines all strings to be used in the interaction with the users (GUI, report, or error messages).
\item The \emph{shape file} specify the detected and shifted bank lines; this format is also used for various debug output.
\item The \emph{image file} stores the figures created by the software; by default the PNG format is used.
\end{itemize}

Each file type is addressed separately in the following subsections.
The shape and image files follow commonly used formats and are hence not explained.

\section{analysis configuration file} \label{Sec:cfg}

The analysis configuration file of WAQBANK looked a lot like an ini-file, but deviated slightly due to different comment style and missing chapter blocks.
\dfastbe still supports the old files, but it's recommended to upgrade them to the newer file format described here which conforms to the usual ini-file format.
The user interface reads old (and obviously new) files, but always writes new files.

The file must contain a \keyw{[General]} block with a keyword \keyw{Version} to indicate the version number of the file.
The initial version number will be \keyw{1.0}.

Details on the other keywords supported are given below.
The order of the keywords in the file doesn't matter, but each block and each keyword should occur only once.

\begin{longtable}{l|l|p{8cm}}
%\begin{tabular}{l|l|p{8cm}}
Block & Keyword & Description \\ \hline
\keyw{General} & \keyw{Version} & Version number. Must be \keyw{1.0} \\
& \keyw{RiverKM} & Name of file with river chainage (kilometres) and corresponding xy-coordinates \\
& \keyw{Boundaries} & River chainage of the region of interest specified as rkm-start:rkm-end, e.g. 81:100 (default: all) \\
& \keyw{BankDir} & Directory for storing bank lines (default: current directory) \\
& \keyw{BankFile} & Name of file(s) in which xy-coordinates of bank lines are stored (default 'bankfile') \\
& \keyw{Plotting} & Flag indicating whether figures should be created (default: true) \\
& \keyw{SavePlots} & Flag indicating whether figures should be saved (default: true) \\
& \keyw{SaveZoomPlots} & Flag indicating whether additional figures should be saved containing zoomed in versions of the standard plots per specified chainage step (default: false) \\
& \keyw{ZoomStepKM} & Chainage step size for the zoom plots (default: 1) \unitbrackets{km} \\
& \keyw{ClosePlots} & Flag indicating whether figures should be closed when the run ends (default: false) \\
& \keyw{FigureDir} & Directory for storing figures (default relative to work dir: figure) \\
& \keyw{FigureExt} & String indicating the file extension and type to be used for the figures saved (default: .png) \\
& \keyw{DebugOutput} & Flag indicating whether additional debug output should be written; this includes various shape and csv files with intermediate results (default: false) \\

\keyw{Detect} & \keyw{SimFile} & Name of simulation output file to be used for determining representative bank line \\
& \keyw{WaterDepth} & Water depth used for defining bank line (default: 0.0) \\
& \keyw{NBank} & Number of bank lines, (default: 2) \\
& \keyw{Line<$i$>} & Name of file with xy-coordinates of search line \keyw{<$i$>} \\
& \keyw{DLines} & Distance from pre-defined lines used for determining bank lines (default: 50) \\

\keyw{Erosion} & \keyw{TErosion} & Simulation period \unitbrackets{years} \\
& \keyw{RiverAxis} & Name of file with xy-coordinates of river axis \\
& \keyw{Fairway} & Name of file with xy-coordinates of fairway axis \\
& \keyw{OutputInterval} & Bin size for which the eroded volume output is given (default: 1) \unitbrackets{km} \\
& \keyw{NLevel} & Number of discharge levels \\
& \keyw{RefLevel} & Reference level: discharge level with \keyw{SimFile<$i$>} that is equal to \keyw{SimFile} (only used when \keyw{Nlevel} > 1)  (default: 1) \\
& \keyw{SimFile<$i$>} & NetCDF map-file for computing bank erosion for discharge \keyw{<$i$>} (only used when \keyw{Nlevel} > 1) \\
& \keyw{PDischarge<$i$>} & Probability of discharge \keyw{<$i$>} (sum of probabilities should be 1) \\
& \keyw{OutputDir} & Directory for storing output files \\
& \keyw{BankNew} & Name of file(s) in which new xy-coordinates of bank lines are stored (default 'banknew') \\
& \keyw{BankEq} & Name of file(s) in which xy-coordinates of equilibrium bank lines are stored (default: 'bankeq') \\
& \keyw{EroVol} & Name of file in which eroded volume per river-km is stored (default: 'erovol.evo') \\
& \keyw{ShipType} & Constant or name of file containing type of ship (per river-km) \\
& \keyw{Vship} & Constant or name of file containing relative velocity of the ships (per river-km) \unitbrackets{m/s} \\
& \keyw{Nship} & Constant or name of file containing number of ships per year (per river-km) \\
& \keyw{Nwave} & Constant or name of file containing number of waves per ship (default 5) \\
& \keyw{Draught} & Constant or name of file containing draught of the ships (per river-km) \unitbrackets{m} \\
& \keyw{Wave0} & Constant or name of file containing distance from fairway axis at which wave height is zero (default 200) \unitbrackets{m} \\
& \keyw{Wave1} & Constant or name of file containing distance from fairway axis at which reduction of wave height to zero starts (default \keyw{Wave0} minus 50) \unitbrackets{m} \\
& \keyw{Classes} & Use classes (true) or critical shear stress (false) in \keyw{BankType} (default: true) \\
& \keyw{BankType} & Constant or \keyw{base} reference to files with bank erosion coefficient definition (for each bank line per river-km) \\
& \keyw{ProtectionLevel} & Constant or \keyw{base} reference to files with level of bank protection for each bank line per river-km (default: -1000) \\
& \keyw{Slope} & Constant or \keyw{base} reference to files with equilibrium slope for each bank line per river-km  (default: 20) \\
& \keyw{Reed} & Constant or \keyw{base} reference to files with reed wave damping coefficient for each bank line per river-km  (default: 0) \\
& \keyw{VelFilterDist} & Specifies the distance/width of a moving average filter for the near bank velocities \unitbrackets{km} (default 0, meaning no filter applied) \\
& \keyw{BedFilterDist} & Specifies the distance/width of a moving average filter for the bank levels \unitbrackets{km} (default 0, meaning no filter applied)
%\end{tabular}
\end{longtable}

\NumNote{When specifying chainage dependent values for \keyw{Slope}, \keyw{Reed}, \keyw{BankType} and \keyw{ProtectionLevel}, please follow the file name conventions listed in \autoref{Sec:parfile}.}

\subsubsection*{Example}

\begin{Verbatim}
[General]
  Version         = 1.0
  RiverKM         = inputfiles\rivkm_20m.xyc
  Boundaries      = 68:230
  BankDir         = files\outputbanklines
  BankFile        = bankline

[Detect]
  SimFile         = inputfiles\SDS-krw3_00-q0075_map.nc
  WaterDepth      = 0.0
  NBank           = 2
  Line1           = inputfiles\oeverlijn_links_mod.xyc
  Line2           = inputfiles\oeverlijn_rechts_mod.xyc
  DLines          = [20,20]

[Erosion]
  TErosion        = 1
  RiverAxis       = inputfiles\maas_rivieras_mod.xyc
  Fairway         = inputfiles\maas_rivieras_mod.xyc
  NLevel          = 2
  RefLevel        = 1
  SimFile1        = inputfiles\SDS-krw3_00-q0075_map.nc
  PDischarge1     = 0.25
  SimFile2        = inputfiles\SDS-krw3_00-q1500_map.nc
  PDischarge2     = 0.75
  OutputDir       = files\outputbankerosion
  BankNew         = banknew
  BankEq          = bankeq
  EroVol          = erovol_standard.evo
  OutputInterval  = 0.1
  ShipType        = 2
  Vship           = 5.0
  Nship           = inputfiles\nships_totaal
  Nwave           = 5
  Draught         = 1.2
  Wave0           = 150.0
  Wave1           = 110.0
  Classes         = false
  BankType        = inputfiles\bankstrength_tauc
  ProtectionLevel = inputfiles\stortsteen
\end{Verbatim}

\section{line geometry files}

This file specifies the coordinates of a single line.
It is used to specify

\begin{itemize}
\item RiverAxis
\item Fairway
\item Initial bank (search) lines
\end{itemize}

The file format is equal to the file format used by WAQBANK.
It consists of two data columns: the first column specifies the x-coordinate and the second column the y-coordinate of each node of the line.

\subsubsection*{Example}

\begin{Verbatim}
1.1887425781300000e+005  4.1442128125000000e+005
1.1888840213863159e+005  4.1442400328947371e+005
1.1890254646426316e+005  4.1442672532894736e+005
1.1891669078989474e+005  4.1442944736842107e+005
1.1893083511552632e+005  4.1443216940789472e+005
1.1894497944115789e+005  4.1443489144736843e+005
1.1895912376678947e+005  4.1443761348684208e+005
1.1897326809242106e+005  4.1444033552631579e+005
1.1898741241805263e+005  4.1444305756578950e+005

...continued...
\end{Verbatim}

Shape files are used for all geospatial output, such as the detected bank lines and the shifted bank lines.

\section{chainage file}

This file defines the river chainage along a line.
The file format is equal to the file format used by WAQBANK.
It consists of three data columns: the first column specifies the chainage, the second and third columns specify the x- and y-coordinates of each node of the line.

\subsubsection*{Example}

\begin{Verbatim}
3     175908.078100      308044.062500
6     176913.890600      310727.781300
10     176886.578100     314661.750000
15     176927.328100     319589.687500
16     176357.000000     320335.375000 

...continued...
\end{Verbatim}

\section{parameter files} \label{Sec:parfile}

These files may be used to define spatial variations in the input parameters needed by the analysis.
Many parameters may be varied along the analyzed river reach.

\begin{itemize}
\item \keyw{wave0}: $s_0$ distance from ship at which the ship waves have reduced to zero \unitbrackets{m}
\item \keyw{wave1}: $s_1$ distance from ship at which the ship waves start to reduce \unitbrackets{m}
\item \keyw{vship}: $v_s$ ship sailing speed \unitbrackets{m/s}
\item \keyw{nship}: $N$ number of empty ships per year \unitbrackets{1/year}
\item \keyw{nwave}: $n$ number of waves per ship \unitbrackets{-}
\item \keyw{draught}: $T_s$ draught of the ship \unitbrackets{m}
\item \keyw{shiptype}: ship type \unitbrackets{-}
\item \keyw{slope}: $\mu_\text{fs}$ wave damping due to sloping foreshore \unitbrackets{m\textsuperscript{-1}} per bank, file extension \keyw{.slp}
\item \keyw{reed}: $\mu_r$ wave damping due to reed \unitbrackets{m\textsuperscript{-1}} per bank, file extension \keyw{.rdd}
\item \keyw{banktype}: bank type \unitbrackets{-} per bank, file extension \keyw{.btp}
\item \keyw{protectionlevel}: top level of the bank protection \unitbrackets{m} per bank, file extension \keyw{.bpl}
\end{itemize}

The file format is independent of the parameter for which it's used.
The file format is equal to the file format used by WAQBANK.
It consists of two data columns: the first column specifies the chainage and the second column the value at that location.
The chainage values should be increasing.
The values are used from the chainage specified on that line until the chainage specified on the next line.
The first value is also used before the chainage specified on the first line.
The last value is used for all chainages larger than the chainage specified on the last line.

\NumNote{Some quantities (\keyw{Slope}, \keyw{Reed}, \keyw{BankType} and \keyw{ProtectionLevel}) need to be specified per bank.
For these parameters the user has to specify a number of files equal to the number of bank lines; for all other parameters only one file is expected that is used for all banks.
The names of the bank dependent files should equal \keyw{base\_1.ext}, \keyw{base\_2.ext}, {\it et cetera} where the user is free to choose \keyw{base} but the extension \keyw{.ext} depends on the quantity selected (see the list above).
The file extension and number part \keyw{\_i} should \emph{not} be included when specifying the file name in the analysis configuration file (see \autoref{Sec:cfg}) or associated user interface field.}

\subsubsection*{Example}

\begin{Verbatim}
65.3    20912
67.6    18529
100.7   20375
146.8   24758
175.5   13911
201     15613
\end{Verbatim}

\section{simulation result files}

For the bank line detection and bank erosion analysis, the program  needs results from a hydrodynamic model.
The WAQBANK program was able to read data from the WAQUA SDS-output files or the Delft3D-FLOW trim-files.
\dfastbe now supports the results of \dflowfm in netCDF map-files following UGRID conventions.
The model needs the following data fields:

\begin{itemize}
\item x-coordinates of the mesh nodes
\item y-coordinates of the mesh nodes
\item face\_node\_connectivity of the mesh
\item bed levels zb at the mesh nodes
\item water levels zw at the mesh faces
\item water depths h at the mesh faces
\item velocity vector (ucx,ucy) at the mesh faces
\end{itemize}

The simulation result files may contain multiple time steps; the final time steps will be used for the analysis.


\section{eroded volume file}

This file reports on the eroded volumes per bank along the analyzed river reach per user defined chainage bin.
The file consists $1+N$ tab-separated data columns where $N$ is the number of bank lines processed: the first column specifies the chainage and the other columns report on the eroded bank volume per bank line accumulated be chainage bin.
The chainage coordinate provided is the upper limit of the chainage bin for which the volume is reported on that line.
The file format differs slightly from the file format used by WAQBANK since that file contained $N$ identical chainage columns followed by the $N$ eroded volume columns.

\subsubsection*{Example}

\begin{Verbatim}
68.00   2.21    0.00
68.10   6.44    0.00
68.20   7.81    0.00
68.30   43.63   161.39
68.40   14.24   0.00
68.50   8.88    0.00
68.60   0.00    0.00
68.70   0.00    0.00
68.80   2.39    0.00
68.90   0.00    0.00
69.00   0.88    0.00
69.10   7.40    69.27
69.20   5.64    65.47
69.30   11.98   55.78

...continued...
\end{Verbatim}

\section{dialog text file}

The dialog text file uses the block labels enclosed by square brackets of the common ini-file format, but the lines in between the blocks are treated verbatim and don't list keyword/value pairs.
Every print statement in the program is associated with a short descriptive identifier.
These identifiers show up in the dialog text file as the block labels.
The text that follows the block label will be used at that location in the program.
The order of the blocks in the file is not important.
Please note that every line is used as is, so don't add indentations or blank lines unless you want those to show up during the program execution.
Most blocks may contain any number of lines, but some blocks may only contain a single line.
Some data blocks may contain one or more Python-style placeholders used for inserting values.

\subsubsection*{Example}

The following excerpt of the default \keyw{messages.UK.cfg} dialog text file shows the string definition for 3 identifiers, namely \keyw{''} (the identifier for an empty line), \keyw{'header\_banklines'} and \keyw{'end\_banklines'}.
The header string contains two placeholders, namely \keyw{\{version\}} for the the version number and \keyw{\{location\}} for the url to the source code location.

\begin{Verbatim}
[]

[header_banklines]
=====================================================
Determine bank lines
=====================================================
version: {version}
source: {location}
-----------------------------------------------------
[end_banklines]

=====================================================
===        BankLines ended successfully!          ===
=====================================================
... continued ...
\end{Verbatim}